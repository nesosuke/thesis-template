% 全体設定
\renewcommand{\rmdefault}{ptm}  % デフォルトフォントの変更: Times New Roman
\renewcommand{\sfdefault}{phv}  % デフォルトフォントの変更: Helvetica
\renewcommand{\ttdefault}{pcr}  % デフォルトフォントの変更: Courier
% ページ設定
\usepackage[top=25mm, bottom=25mm,left=30mm, right=15mm, includefoot, footskip=15mm]{geometry}
\def\linesparpage#1{
    \baselineskip=\textheight
    \divide\baselineskip by #1
} % 1ページあたりの行数
\linesparpage{40} % 1ページあたりの行数
% 目次関連
\renewcommand{\cfttoctitlefont}{\mcfamily\rmfamily} % 目次のタイトルのフォント
\renewcommand{\cftsecfont}{\mcfamily\rmfamily} % 目次におけるセクションのフォント
\renewcommand{\cftsubsecfont}{\mcfamily\rmfamily} % 目次におけるサブセクションのフォント
\renewcommand{\cftsubsubsecfont}{\mcfamily\rmfamily} % 目次におけるサブサブセクションのフォント
\renewcommand{\cftsecaftersnum}{.} % 目次におけるセクションの番号の後のドット
\renewcommand{\cftsubsecaftersnum}{.} % 目次におけるサブセクションの番号の後のドット
\renewcommand{\cftsubsubsecaftersnum}{.} % 目次におけるサブサブセクションの番号の後のドット
\renewcommand{\cftsecdotsep}{2} % 目次におけるセクションのドットの間隔
\renewcommand{\cftsubsecdotsep}{2} % 目次におけるサブセクションのドットの間隔
\renewcommand{\cftsubsubsecdotsep}{2} % 目次におけるサブサブセクションのドットの間隔
% 見出し関連
\titleformat*{\section}{\large} % セクション見出しのフォントサイズ
\titleformat*{\subsection}{\normalsize} % サブセクション見出しのフォントサイズ
\titleformat*{\subsubsection}{\normalsize} % サブサブセクション見出しのフォントサイズ
% 図表関連
\renewcommand{\figurename}{図} % 図番号を「図」に変更
\renewcommand{\tablename}{表} % 表番号を「表」に変更
% 文献関連
\makeatletter
\renewcommand{\@cite}[1]{\textsuperscript{(#1)}} % 参照文献の番号を上付きに変更
\makeatother
